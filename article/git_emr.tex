% THIS IS SIGPROC-SP.TEX - VERSION 3.1
% WORKS WITH V3.2SP OF ACM_PROC_ARTICLE-SP.CLS
% APRIL 2009
%
% It is an example file showing how to use the 'acm_proc_article-sp.cls' V3.2SP
% LaTeX2e document class file for Conference Proceedings submissions.
% ----------------------------------------------------------------------------------------------------------------
% This .tex file (and associated .cls V3.2SP) *DOES NOT* produce:
%       1) The Permission Statement
%       2) The Conference (location) Info information
%       3) The Copyright Line with ACM data
%       4) Page numbering
% ---------------------------------------------------------------------------------------------------------------
% It is an example which *does* use the .bib file (from which the .bbl file
% is produced).
% REMEMBER HOWEVER: After having produced the .bbl file,
% and prior to final submission,
% you need to 'insert'  your .bbl file into your source .tex file so as to provide
% ONE 'self-contained' source file.
%
% Questions regarding SIGS should be sent to
% Adrienne Griscti ---> griscti@acm.org
%
% Questions/suggestions regarding the guidelines, .tex and .cls files, etc. to
% Gerald Murray ---> murray@hq.acm.org
%
% For tracking purposes - this is V3.1SP - APRIL 2009

\documentclass{acm_proc_article-sp}
\usepackage{hyperref}

\begin{document}

\title{GitEMR: A Distributed Version Control System for Electronic Medical Records}%\titlenote{}}
%\subtitle{ \titlenote{} }
%
% You need the command \numberofauthors to handle the 'placement
% and alignment' of the authors beneath the title.
%
% For aesthetic reasons, we recommend 'three authors at a time'
% i.e. three 'name/affiliation blocks' be placed beneath the title.
%
% NOTE: You are NOT restricted in how many 'rows' of
% "name/affiliations" may appear. We just ask that you restrict
% the number of 'columns' to three.
%
% Because of the available 'opening page real-estate'
% we ask you to refrain from putting more than six authors
% (two rows with three columns) beneath the article title.
% More than six makes the first-page appear very cluttered indeed.
%
% Use the \alignauthor commands to handle the names
% and affiliations for an 'aesthetic maximum' of six authors.
% Add names, affiliations, addresses for
% the seventh etc. author(s) as the argument for the
% \additionalauthors command.
% These 'additional authors' will be output/set for you
% without further effort on your part as the last section in
% the body of your article BEFORE References or any Appendices.

\numberofauthors{2} %  in this sample file, there are a *total*
% of EIGHT authors. SIX appear on the 'first-page' (for formatting
% reasons) and the remaining two appear in the \additionalauthors section.
%
\author{
% You can go ahead and credit any number of authors here,
% e.g. one 'row of three' or two rows (consisting of one row of three
% and a second row of one, two or three).
%
% The command \alignauthor (no curly braces needed) should
% precede each author name, affiliation/snail-mail address and
% e-mail address. Additionally, tag each line of
% affiliation/address with \affaddr, and tag the
% e-mail address with \email.
%
% 1st. author
\alignauthor
Kathleen Ruroede\titlenote{Kathleen serves as Vice President, Quality 
and Research and holds the titles PhD, MEd, RN}\\
       \affaddr{Marianjoy Rehabilitation Hospital}\\
       \affaddr{26W171 Roosevelt Road}\\
       \affaddr{Wheaton, IL 60187}\\
       \email{kruroede@marianjoy.org}
% 2nd. author
\alignauthor
Kurt Rudolph\\
       \affaddr{University of Illinois Urbana-Champaign, Department of Computer Science}\\
       \affaddr{201 North Goodwin Avenue}\\
       \affaddr{Urbana, IL 61801-2302}\\
       \email{rudolph9@illinois.edu}
}
% There's nothing stopping you putting the seventh, eighth, etc.
% author on the opening page (as the 'third row') but we ask,
% for aesthetic reasons that you place these 'additional authors'
% in the \additional authors block, viz.

%\additionalauthors{Additional authors: John Smith (The Th{\o}rv{\"a}ld Group,
%email: {\texttt{jsmith@affiliation.org}}) and Julius P.~Kumquat
%(The Kumquat Consortium, email: {\texttt{jpkumquat@consortium.net}}).}
%\date{30 July 1999}

% Just remember to make sure that the TOTAL number of authors
% is the number that will appear on the first page PLUS the
% number that will appear in the \additionalauthors section.

\maketitle
\begin{abstract}
This paper describes the applicability of \textit{distributed version control} to 
\textit{electronic medical records}. Demand for more efficient and effective methods
of managing \textit{electronic medical records} is growing.  
While \textit{distributed database systems} have been successfully utilized in leisure 
oriented applications they generally do not guarantee data integrity and in turn are 
not suited for \textit{electronic medical records}.  
\textit{Distributed version control} systems are designed for collaborative modification of electronic
data while maintaining data integrity and may serve an efficient and effective method 
of managing \textit{electronic medical records}.
\end{abstract}

% A category with the (minimum) three required fields
\category{E.1}{ Data Structures}{distributed data structures, records}
%A category including the fourth, optional field follows...
\category{H.2.4}{Database Management}{Systems}[distributed databases, concurrency]

\terms{}

\keywords{Electronic Medical Records, Distributed Database Systems}

\section{Introduction}
\textit{Electronic medical records} (\textit{EMRs}) have recently begun their adolescence.  
Current applications managing the data of EMRs have demonstrated the benefits
of storing medical records electronically and offer insight into the 
applicability of distributed version control to EMRs.

More mature methods for managing \textit{EMRs} are necessary for the continued advancement of
medical software applications.  Currently employed methods limit the applicability of medical 
data analysis, hinder collaborative modification, and require unnecessary costs.

Features offered by the Git version control system are suiting for the management of \textit{EMRs}.
Throughout this paper the applicability of distributed version control will be discussed and specifically,
a implementation  proposal for GitEMR will be defined.

\subsection{Audience}
Medical professionals, and in particular those tasked with aggregation and analysis of medical data.

\subsection{Related Work}

\href{http://www.open-emr.org/}{OpenEMR}


\subsection{Structure of this Document}
The article begins by outlining related topics generally not familiar to medical professionals.


\section{Database Systems}

\subsection{Distributed Database Systems}
\subsection{Version Control Systems}
\subsubsection{Versioning Data}
\subsubsection{Collaboration}
\subsection{Distributed Version Control Systems}
\subsubsection{Versioning Data}
\subsubsection{Collaboration}
\subsubsection{The Repository}
\subsubsection{Branches and Tags}
\subsubsection{Working Trees and Working Bases}
\subsubsection{Commits and Revisions}
\subsubsection{Merging, Pushing, and Pulling}

\section{Electronic Medical Records}
\subsection{Current Employed Models}
\subsubsection{Cost Effectiveness}

storage, maintenance, accessibility, and acquisition

Although the price per digital storage unit is rapidly decreasing, the methods for storing 
the data of EMRs is generally very costly in caparison to methods employed by comparable 
applications.  

The level of technical expertise is generally much higher than necessary...

The accessibility of the 

\subsection{Confidentiality}

\section{Applying Git Version Control to Electronic Medical Records}

\subsection{Applicability}
\subsection{Issues}



\section{Random stuff}

EMRs generally maintain extremely sensitive data.  Accuracy of all data
maintained by an EMR potentially determines the survival of the individual a record 
reflects.  Access to data maintained by an EMR is generally restricted, only accessible 
to those authorized the individual a reflects.

Applications are generally costly in terms of initial acquisition, continued maintenance, 
and accessibility.




This Applying distributed database systems to applications which manage the data of EMRs
potentially offers a more efficient and effective method 

creating, reading, and updating are needed.



\end{document}
